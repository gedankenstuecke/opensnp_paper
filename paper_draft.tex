% Template for PLoS
% Version 1.0 January 2009
%
% To compile to pdf, run:
% latex plos.template
% bibtex plos.template
% latex plos.template
% latex plos.template
% dvipdf plos.template

\documentclass[10pt]{article}

% amsmath package, useful for mathematical formulas
\usepackage{amsmath}
% amssymb package, useful for mathematical symbols
\usepackage{amssymb}

% graphicx package, useful for including eps and pdf graphics
% include graphics with the command \includegraphics
\usepackage{graphicx}

% cite package, to clean up citations in the main text. Do not remove.
\usepackage{cite}

\usepackage{color} 

% Use doublespacing - comment out for single spacing
%\usepackage{setspace} 
%\doublespacing


% Text layout
\topmargin 0.0cm
\oddsidemargin 0.5cm
\evensidemargin 0.5cm
\textwidth 16cm 
\textheight 21cm

% Bold the 'Figure #' in the caption and separate it with a period
% Captions will be left justified
\usepackage[labelfont=bf,labelsep=period,justification=raggedright]{caption}

% Use the PLoS provided bibtex style
\bibliographystyle{plos2009}

% Remove brackets from numbering in List of References
\makeatletter
\renewcommand{\@biblabel}[1]{\quad#1.}
\makeatother


% Leave date blank
\date{}

\pagestyle{myheadings}
%% ** EDIT HERE **


%% ** EDIT HERE **
%% PLEASE INCLUDE ALL MACROS BELOW

%% END MACROS SECTION

\begin{document}

% Title must be 150 characters or less
\begin{flushleft}
{\Large
\textbf{openSNP - Crowdsourcing Genome-Wide Association Studies}
}
% Insert Author names, affiliations and corresponding author email.
\\
Bastian Greshake$^{1,\ast}$, 
Philipp Bayer$^{2}$, 
Fabian Zimmer$^{3}$,
Julia Reda$^{4}$
\\
\bf{1} Goethe University, Frankfurt am Main, Germany
\\
\bf{2} Bond University, Gold Coast, Australia
\\
\bf{3} Westf\"alische Wilhelms Universit\"at, M\"unster, Germany
\\
\bf{4} Johannes-Gutenberg University, Mainz, Germany
\\
$\ast$ E-mail: info@opensnp.org
\end{flushleft}

% Please keep the abstract between 250 and 300 words
\section*{Abstract}
Genome-wide association studies are a cheap and quick way of assessing health risks by comparing Single Nucleotide Polymorphisms between groups of participants. Direct-To-Consumer Companies like 23andme offer SNP sequencing to their customers alongside with an evaluation of the customer's genetic risks. However, the data 23andme and other companies generate is not accessible to other scientists, and withholds some information from customers for various reasons. In this paper, we present an open approach to Genome-wide association studies by introducing openSNP, a web platform which allows customers to openly share their polymorphisms with scientists for free.  % misses survey


% Please keep the Author Summary between 150 and 200 words
% Use first person. PLoS ONE authors please skip this step.
% Author Summary not valid for PLoS ONE submissions.   
\section*{Author Summary}

\section*{Introduction}

Genome-Wide Association Studies (GWAS) are a comparatively easy and cheap way of finding Single Nucleotide Polymorphisms (SNPs) of medical relevance. Such SNPs of interest can be used to find candidate genes for a closer inspection or to predict disease risks or other traits. Genome-Wide Association Studies make use of statistics for comparing the alleles of patients to the alleles of healthy controls. This method does not allow inference of causal differences but merely identifies correlations. The first GWAS was published in 2005 and compared age-related macular degeneration in contrast to a healthy control group \cite{Klein2005}. Since the beginning, the number of participants in such studies has been rising: Over 1200 GWAS have been performed \cite{Johnson2009} and over 5000 SNPs have been linked to different diseases and traits in those studies \cite{Hindorff2009}. %(http://www.genome.gov/page.cfm?pageid=26525384&clearquery=1#result_table)

Since 2006, companies like 23andMe, deCODEme or FamilyTreeDNA have been offering Direct-To-Consumer (DTC) genetic testing. Those companies use DNA microarrays to screen for around 1 million SNPs spread over the human genome. In return customers get an analysis of the results, as well as a raw file that includes the customer's individual genotypes. In 2011, 23andMe alone had over 100,000 customers\footnote{http://spittoon.23andme.com/2011/06/15/23andme-2011-state-of-the-database-address/} - the company recognizes the potential to perform GWAS with this amount of data by using surveys to ask their customers about traits and diseases. With the consent of the customer the data is used for association studies. 23andMe has published several articles in which it replicates known findings, but also finds new associations for Parkinson's Disease \cite{Eriksson2010, Do2011}. Over 30,000 23andme-customers have participated in those association studies.  

Although companies like 23andMe are willing to contribute to science, it is not easy for individual scientists to access the data. This is mainly due to concerns about privacy, liability and consent. Nevertheless, there are individual customers who are willingly sharing their data. Most do so by uploading their data to their personal website or to open software repositories like \textit{GitHub}. While this makes it possible for scientists to access the data, it requires a lot of work to keep track of all new genotyping data that is available to the public. While projects like the SNPedia try to keep track of all the files \cite{Cariaso2011}, they do not provide the information necessary to perform GWAS, as the phenotypic information is not attached to the genetic information. Projects that attach the phenotype to the genetic information, like the \textit{Personal Genome Project}, still don't allow for an easy re-use of the data, as they lack an advanced programming interface (API) or other methods by which researchers could download the data.  

A possible solution to this is a community-driven platform that aggregates genetic and phenotypic information of people who are willing to share their data with the general public and have given their informed consent. We designed an online survey to assess interest in such a crowdsourcing platform, in which we asked an audience interested in biology topics about their willingness to share their genetic and phenotypic information with the public. The aim of the survey was not to determine the percentage of the population interested in sharing genotypic information, but to get an idea of whether a crowdsourcing platform could get enough support to go forward with the project. Additionally, we built a platform which allows customers of DTC genetic testing to publicize genetic and phenotypic information and gives researchers multiple ways of reusing the data. 

% Results and Discussion can be combined.
\section*{Results}

\subsection*{Survey on Sharing Genetic Information}
229 people, 180 with a self-reported chromosomal sex of XY, 56 with a self-reported chromosomal sex of XX, participated in the survey. The mean age of the participants is 33 (SD = 11,29) and 81.7 \% reported their ethnicity as caucasian. 39.7 \% of the participants are already customers of at least one DTC genetic testing company and further 30.1 \% of them plan on becoming one in the future. 29.7 \% don't plan on becoming a DTC customer. There is no significant difference in the usage of DTC companies between chromosomal sexes (Somers-d). 

67.7 \% of all participants would share their data with their DTC-company without any constraints, 25.8 \% would do so given the company didn't share the data with third parties. 6.6 \% of the participants would not share their data. There is no significant difference in readiness to share between chromosomal sexes (Somers-d). Those who are a customers of a DTC company or are planning on becoming one in the future are more likely to share their results, compared to those who don't plan on getting themselves genotyped (Somers-d). 

There are significant differences in terms of motivation, tested by Tukey's HSD test, between those people who have already been genotyped and those who are not planning on getting genotyped. The first group is likely to agree more strongly, on a five-point scale, with the following motivations for sharing genotypic information: because they want to help scientists (mean difference = 0.465, SE = 0.128, p = 0.001), because they think of personal benefits (mean difference = 0.448, SE = 0.183, p = 0.04) and because they are curious (mean difference = 1.159, SE = 0.193, p < 0.001). 

On the other hand, those people who are not planning on getting genotyped are more likely to agree with the following motivations for not sharing their data: because they fear discrimination (mean difference = 1.060, SE = 0.195, p < 0.001), because they feel it is a breach of their privacy (mean difference = 0.821, SE = 0.225, p = 0.001), because they fear negative consequences for their family members (mean difference = 0.733, SE = 0.21, p = 0.002) or because they fear personalized advertising (mean difference = 0.848, SE = 0.208, p < 0.001).

Similarly, those people who would share data with their DTC provider under any circumstances are likely to agree more strongly with the following motivations for sharing than those who would not share their data with their DTC company: because they want to help scientists (mean difference = 1.57, SE = 0.199, p < 0.001), because they think of personal benefits (mean difference = 0.951, SE = 0.308, p = 0.006), and because they are curious (mean difference = 1.99, SE = 0.321, p < 0.001). 

Those participants who are not willing to share data with their DTC company are likely to agree more strongly with the following motivations for not sharing their data when compared to those who would share their data with their DTC company under any circumstances: because they fear discrimination (mean difference = 1.52, SE = 0.322, p < 0.001), because they feel it is a breach of their privacy (mean difference = 1.871, SE = 0.324, p < 0.001), because they fear consequences for their familiy members (mean difference = 1.146, SE = 0.32, p = 0.001) and because they fear personalized advertising (mean difference =  1.112, SE = 0.357, p = 0.006). 

\subsection*{openSNP}
We created openSNP, a website which allows users to upload their genotypings from the companies 23andme, deCODEme and Family Tree under the Creative Commons Zero-license, which - in accordance with the Panton Principles (doi:10.1371/journal.pbio.1001195) - allows a complete reuse of the data without any constraints. Users are encouraged to list as many phenotypes as possible through a simple achievement system which rewards users that upload their data and enter phenotypic information with small badges that are shown on their profile pages. 

The specification of variations on a single phenotype and the phenotypes themselves is open and not limited to pre-defined categories. Every user can add completely new phenotypes that are not yet queried by openSNP. To reduce the amount of manual data curation, openSNP tries to harmonize the expression and spelling of the same phenotype or variation by implementing an autocompletion-feature which lists similar entries to what is being entered that are already in the openSNP database.

OpenSNP offers extensive access to the data uploaded by users. Anyone can download single genotyping files for specific users, get archives of multiple genotyping files grouped by phenotypic variation, or access a single download that includes all genotyping files and all phenotypes in a comma-separated table. Additionally, the genetic data is accessible through the Distributed Annotation System, which offers all data for specific chromosomes and specific positions on single chromosomes. Users can discuss SNPs and phenotypes on the platform using a simple commenting system or private messages. 

Between the start of openSNP on 09/27/2011 and 12/18/2011, 214 people have signed up with openSNP, 79 of which have uploaded their genotyping files. The openSNP database thereby lists 69,486,471 SNPs, which are distributed over 1,938,604 unique Rs-IDs. In the same timeframe all users combined have entered a total of 675 variations on 47 different phenotypes. See figure n for a distribution of data acquisition over time.

The mean number of users that have entered their variations for a single phenotype is 14.36 (SD 12.65), the median is 10. The distribution of how many users have entered their data per phenotype can be seen in figure n+1. The phenotype provided by the most users is the eye color, which has been entered by 54 users. There are two phenotypes which have so far only been provided by a single user: The SAT Writing score and triglyceride-levels.

In order to provide users with relevant information on their respective genotypes, openSNP scans databases of the scientific literature for specific SNPs. A total number of 15,229 documents relevant to the SNP-IDs listed in openSNP could be found in the databases of Mendeley, the Public Library of Science and SNPedia. Of the primary literature, 25 \% are released in open access journals and can be accessed free of charge by every user (Figure n+2). For usability reasons, SNPs are ranked by the amount of information gathered by the external services.

The external services themselves are ranked by how easily users can access information from these sources. The SNPedia entries are given the highest impact, as those are already manually curated, followed by open access publications out of the Public Library of Science. Lowest values are given to the Mendeley results, as those aren't necessary freely available to every user. An entry on SNPedia is valued 2.5 times as high as a PLoS publication and 5 times as high as a Mendeley entry.  

\section*{Discussion}
Although prices of exome or even full genome sequencing are dropping rapidly, GWAS tend to stay considerably cheaper and offer a possibility of getting insight into genetic variations on a population level and allow an analysis of SNPs which are linked to different traits. Despite this benefit in terms of cost, it must be pointed out that GWAS can only detect correlations of SNPs with those traits and don't allow inference on the cause for any correlation. Furthermore, GWAS need a large enough sample which has to be statistically analysed by using sound methods. Nevertheless, GWAS are still frequently used and new associations still can be found (we should list some recent results here -> oxytocin maybe? we could look at the twitter-timeline of @opensnporg, i publish papers i find there).

One way of bringing down costs for GWAS even further is to make use of already available genotyping results and datasets. Data produced by DTC genetic testing companies is a promising source for such data, as such companies already have high numbers of customers which are willing to pay for the genotyping by themselves. As we have found in the survey on sharing such results, many of these customers are also willing to share their results with scientists and the public to help scientific progress, although people who have taken DTC genetic testing are aware of the privacy implications that come with openly sharing those results.      

With openSNP, we have built a platform that can be used by customers of DTC genetic testing to easily share their genetic and phenotypic data with a wide audience, as well as by scientists and interested citizens who are looking for datasets to be used in GWAS. Customers of DTC genetic testing also benefit from an easy access to primary literature on SNPs and genetic variations they carry. While we don't have collected enough data to perform full scale GWAS yet, this might be possible in the future, as user numbers are rising. 

By crowdsourcing the acquisition of genetic and phenotypic data, openSNP faces the same problems as any other open platform on the Internet, namely the need to trust users regarding the data they upload and enter on openSNP. Additionally, the quality of the data varies, especially in terms of accuracy on the phenotypic variation. While we try to suggest similar entries to the users, there are some cases where users wont follow those suggestions, so duplicates or similar phenotypes or varations in traits may arise. 

There are two possible solutions to this problem: One could only allow some (trusted) users to enter new phenotypes or one could make users enter all possible variations of a phenotype while creating a new phenotype, so that later users can't add variations that have not been available from the start. Both methods have a disadvantage: In either case it makes it harder for users to enter their data and thereby raises the bar for participation, which ultimately could lead to less data entered. Facing this trade-off, we decided to keep data entry as easy as possible, at the cost of forcing users who want to perform GWAS with the data to perform more quality control.

Another risk regarding data quality that should be kept in mind is a possible bias in data availability on openSNP, as we cannot rule out the possibility that only an unrepresentative subset of people buy DTC genetic testing and an even smaller subset is willing to publish the results, along with information of medical relevance.

The advent of DTC genetic testing has led to new ethical and social issues. Much of the critique on DTC genetic testing focusses on the practice of delivering medical information without consulting a physician or genetic counselor to help patients/customers make sense of the information and to put the new knowledge to good use \cite{Hauskeller2011,Hogarth2008,Wasson2009}. Less attention has been paid to the privacy implications that come with this cheap way of obtaining genetical information \cite{Caulfield2011,Joh2011}. 

With a move towards making one's own genetic and medical information public, it becomes clear that the issue of privacy needs to get a closer look. Our survey has shown that people are concerned about their privacy and fear that stakeholders like employers, insurance companies, governments or advertisers might misuse the information. Policy makers start to react to those changes by introducing laws like the \textit{Genetic Information Non-Discrimination Act} in the United States or the \emph{Gendiagnostikgesetz} in Germany to minimize the impact of widely available genetic information. DTC genetic testing companies themselves also try to educate their customers about the risks of releasing genetic data.  

We openly address the problem of privacy implications that come with releasing genetic data twice, during registration for openSNP and during the upload of the DTC genetic testing results. Users have to confirm that they have read and understood the disclaimer about possible side-effects that can arise by publishing their data. To further improve this process and get informed consent, we are working on implementing the procedures which are currently developed by \textit{Consent for Research} (http://www.weconsent.us).
 
% You may title this section "Methods" or "Models". 
% "Models" is not a valid title for PLoS ONE authors. However, PLoS ONE
% authors may use "Analysis" 
\section*{Materials and Methods}
\subsection*{Survey on Sharing Genetic Information}
The survey was done with Google Docs and was distributed to possible participants through the 23andMe community, the DIYBiology mailing list, blogs which focus on genetics and DTC genetic testing and social media like Twitter, Google Plus and Facebook. Because of this approach, the results do not reflect the general population, but overrepresent those people most likely to be interested in a project such as openSNP: customers of DTC genetic testing companies and people with a high interest in biology. As we wanted to find out about interest in crowdsourcing GWAS and about how people who had purchased DTC genetic testing felt about sharing their data with third parties, this approach was warranted. 

The survey included demographics such as age, chromosomal sex and ethnicity of the participants. Furthermore, it included questions on their (planned) customership of a DTC company. If they already were customers, they were also asked if they were already sharing their genetic and phenotypic data. All participants were asked if they would share their genetical or phenotypic information with their DTC company, possible answers were "Yes", "Yes, but only if they did not share my medical information with anybody else" and No".

The survey also asked some scaled questions which measured how strongly participants agreed/disagreed with different reasons for sharing or not sharing their information with third parties. The scale went from 1 = strongly disagree over 3 = neutral to  5 = strongly agree. Motivations queried for sharing data were "because I want to help scientists with their research", "because of possible personal benefits (e.g. getting treatments for a disease I have, possibility of new medication, etc.)", "because it may deliver advertising that is relevant to me" and "out of curiosity". Motivations queried for not sharing data were "because advertisers could use the information for targeted campaigns", "because of possible negative consequences for closely related persons", "because of the breach of my privacy" and " because of the fear of discrimination (e.g. by the employer, the state, some insurance company)". Additionally, participants had the possibility of giving their own reasons for sharing or not sharing their data.

The survey data was analyzed with SPSS 19. 

\subsection*{Technical realization of the platform}
The main platform is realized using the web framework Ruby on Rails. Postgres is used as the main database backend for Rails. The database stores genotyping results, users' phenotypic information, literature results from Mendeley and the Public Library of Science as well as summaries on SNPs which can be found in SNPedia. The literature database of Mendeley is queried using the REST API, which delivers results in JSON. The literature database of the Public Library of Science is queried using the respective REST API, which delivers results in an XML-format. Summaries on SNPs are provided by SNPedia, through querying the content via the MediaWiki API. All databases are queried using the unique identifier of each SNP as the search term. 

SNPs are cataloged by their unique identifier, which consists of a prefix (mostly \textit{rs}) and a unique number. This is a common format, which is employed by the NCBI dbSNP database and is also widely used and easily parsed from different literature sources. Publications from the different databases as well as the users' genotypes are associated with individual SNPs by the Rs-ID. Allele and genotype frequencies are updated regularly, based on the data present in openSNP. 

Processes with a longer runtime, such as parsing the genotyping results, creating archives of results which are to be mailed to users and queries to external resources are handled using the ruby gem Resque and a Redis server. Search features on the platform itself are implemented using SOLR and the ruby gem Sunspot. Additionally, data can be requested from openSNP using the Distributed Annotation System \cite{Dowell2001,Jenkinson2008}. The required data is stored in a mySQL database, the delivery of the data is done by using ProServer \cite{Finn2007}, which was modified by Gel et al. for use in easyDAS \cite{GelMoreno2011}.  

A flowchart of all services incorporated in openSNP and of all the ways users can upload or access the data is given in figure n+3. The source code of openSNP is published under Creative Commons BY-SA 3.0 and can be downloaded at http://github.com/gedankenstuecke/snpr. The genetical and phenotypical data is licensed under Creative Commons Zero. 
% Do NOT remove this, even if you are not including acknowledgments
\section*{Acknowledgments}


%\section*{References}
% The bibtex filename
\bibliography{papers}

\section*{Figure Legends}
%\begin{figure}[!ht]
%\begin{center}
%%\includegraphics[width=4in]{figure_name.2.eps}
%\end{center}
%\caption{
%{\bf Bold the first sentence.}  Rest of figure 2  caption.  Caption 
%should be left justified, as specified by the options to the caption 
%package.
%}
%\label{Figure_label}
%\end{figure}


\section*{Tables}
%\begin{table}[!ht]
%\caption{
%\bf{Table title}}
%\begin{tabular}{|c|c|c|}
%table information
%\end{tabular}
%\begin{flushleft}Table caption
%\end{flushleft}
%\label{tab:label}
% \end{table}

\end{document}