% Template for PLoS
% Version 1.0 January 2009
%
% To compile to pdf, run:
% latex plos.template
% bibtex plos.template
% latex plos.template
% latex plos.template
% dvipdf plos.template

\documentclass[10pt]{article}

% amsmath package, useful for mathematical formulas
\usepackage{amsmath}
% amssymb package, useful for mathematical symbols
\usepackage{amssymb}

% graphicx package, useful for including eps and pdf graphics
% include graphics with the command \includegraphics
\usepackage{graphicx}

% cite package, to clean up citations in the main text. Do not remove.
\usepackage{cite}

\usepackage{color} 

% Use doublespacing - comment out for single spacing
%\usepackage{setspace} 
%\doublespacing

%FZ: Enable the comment command
\usepackage{verbatim}
% Text layout
\topmargin 0.0cm
\oddsidemargin 0.5cm
\evensidemargin 0.5cm
\textwidth 16cm 
\textheight 21cm

% Bold the 'Figure #' in the caption and separate it with a period
% Captions will be left justified
\usepackage[labelfont=bf,labelsep=period,justification=raggedright]{caption}

% Use the PLoS provided bibtex style
\bibliographystyle{plos2009}

% Remove brackets from numbering in List of References
\makeatletter
\renewcommand{\@biblabel}[1]{\quad#1.}
\makeatother


% Leave date blank
\date{}

\pagestyle{myheadings}
%% ** EDIT HERE **


%% ** EDIT HERE **
%% PLEASE INCLUDE ALL MACROS BELOW
\long\def\authornote#1{%
        \leavevmode\unskip\raisebox{-3.5pt}{\rlap{$\scriptstyle\diamond$}}%
        \marginpar{\raggedright\hbadness=10000
        \def\baselinestretch{0.8}\tiny
        \it #1\par}}
\newcommand{\bastian}[1]{\authornote{BG: #1}}
\newcommand{\fabian}[1]{\authornote{FZ: #1}}
\newcommand{\philipp}[1]{\authornote{PB: #1}}
%% END MACROS SECTION
\begin{document}

% Title must be 150 characters or less
\begin{flushleft}
{\Large
\textbf{openSNP - Crowdsourcing Genome-Wide Association Studies - Supplementary Methods}
}
% Alternative titles: 
% openSNP - a new, open data-source for personalised medicine
% What kind of person would share genotyping-data? Presenting a survey and an open data-source for personalised medicine
% 
% Insert Author names, affiliations and corresponding author email.
\\
Bastian Greshake$^{1,\ast}$, 
Philipp Bayer$^{2}$, 
Fabian Zimmer$^{3}$,
Julia Reda$^{4}$
\\
\bf{1} Goethe University, Frankfurt am Main, Germany
\\
\bf{2} University of Queensland, Brisbane, Australia
\\
\bf{3} Westf\"alische Wilhelms Universit\"at, M\"unster, Germany
\\
\bf{4} Johannes-Gutenberg University, Mainz, Germany
\\
$\ast$ E-mail: info@opensnp.org
\end{flushleft}

\section*{Introduction}
To evaluate whether customers of Direct-To-Consumer (DTC) genetic testing would be willing to share their raw data, along with phenotypic annotations, freely on the web a survey on the topic was set up and spread among customers and potential customers of DTC testing. 

\section*{Materials and Methods}
\subsection*{Survey on Sharing Genetic Information}
The survey was performed using \textit{Google Docs} and was distributed to possible participants through the \textit{23andMe }community forums, the \textit{DIYBiology} mailing list, 
blogs which focus on genetics and DTC genetic testing and social media like \textit{Twitter}, \textit{Google+} and \textit{Facebook}.  

The survey included demographics such as age, chromosomal sex and ethnicity of the participants. Furthermore, it included questions on their 
(planned) customer-ship with a DTC company. If the participants already were customers, they were also asked if they were already sharing their genetic and phenotypic data. 
All participants were asked if they would share their genetical or phenotypic information with their DTC company, possible answers were "Yes", "Yes, 
but only if they did not share my medical information with anybody else" and No".

The survey also asked some scaled questions which measured how strongly participants agreed/disagreed with different reasons for sharing or not sharing their 
information. The scale went from 1 = strongly disagree to  5 = strongly agree. Motivations queried for sharing data 
were "because I want to help scientists with their research", "because of possible personal benefits (e.g. getting treatments for a disease I have, 
possibility of new medication, etc.)", "because it may deliver advertising that is relevant to me" and "out of curiosity". Motivations queried for not sharing 
data were "because advertisers could use the information for targeted campaigns", "because of possible negative consequences for closely related persons", 
"because of the breach of my privacy" and "because of the fear of discrimination (e.g. by the employer, the state, some insurance company)". 
Additionally, participants had the possibility of giving their own reasons for sharing or not sharing their data.

The survey data was analyzed with SPSS 19. 

% Results and Discussion can be combined.
\section*{Results}

\subsection*{Survey on Sharing Genetic Information}
In total 229 people, 180 with a self-reported chromosomal sex of XY and 56 with a self-reported chromosomal sex of XX participated in our survey on sharing genetic information with the public. 
The mean age of the participants is 33 (SD = 11,29). 81.7 \% reported their ethnicity as caucasian. 39.7 \% of the participants are already 
customers of at least one DTC genetic testing company and further 30.1 \% of them plan on becoming one in the future. 29.7 \% do not plan on 
becoming a DTC customer. There is no significant difference in the usage of DTC companies between chromosomal sexes (Cramer's V = 0.077). 

67.7 \% of all participants would share their data with their DTC-company without any constraints, 25.8 \% would do so given the company 
didn't share the data with third parties. 6.6 \% of the participants would not share their data. Participants self-identified as XX-chromosomal are slightly more likely to answer that DTC companies are allowed to use their results (Cramer's V = 0.221). Those who are customers of a DTC company or are planning on becoming one in 
the future are more likely to share their results, compared to those who do not plan on getting themselves genotyped (Somers-d = 0.331). 


There are substantial differences in terms of motivation, tested by Tukey's HSD test, between those people who have already been genotyped 
and those who are not planning on getting genotyped. The first group is likely to agree more strongly, on a five-point scale, with motivations for sharing genotypic information. On the other hand, those people who are not planning on getting genotyped are more likely to agree with the following motivations 
for not sharing their data, see table \ref{tab:motivations1}.

\begin{table}
\begin{tabular}{|l|l|l|}
\hline
& Turkey's HSD & \\ \cline{2-3}
& Mean difference & SE \\ \hline
\textbf{Motivation for sharing genotypings in participants} & & \\
\textbf{who are already genotyped} & & \\ \hline 
... curious & 1.159 & 0.193 \\ 
... want to help scientists & 0.465 & 0.128 \\
... for personal benefits & 0.448 & 0.183 \\ \hline
\textbf{Motivation for not sharing in participants} & & \\ 
\textbf{who are not planning to get genotyped} & & \\ \hline
... fear of discrimination & 1.06 & 0.195 \\
... breach of privacy & 0.821 & 0.225 \\
... fear of personalized advertising & 0.848 & 0.208 \\ 
... negative consequences for family members & 0.733 & 0.21 \\ \hline 
\end{tabular}
\caption{Differences in terms of motivation to share genotypings with the public in survey-participants who already received a genotyping compared to participants who are not planning to getting genotyped. }
\label{tab:motivations1}
\end{table}

\begin{table}
\begin{tabular}{|l|l|l|}
\hline
& Turkey's HSD & \\ \cline{2-3}
& Mean difference & SE \\ \hline
\textbf{Motivation for sharing genotypings in participants} & & \\
\textbf{who would share with their DTC provider} & & \\ \hline
... curiosity & 1.99 & 0.321 \\
... want to help science & 1.57 & 0.199 \\
... for personal benefits & 0.951 & 0.308 \\ \hline
\textbf{Motivation for sharing genotypings in participants} & & \\
\textbf{who would not share with their DTC provider} & & \\ \hline
... fear of discrimination & 1.52 & 0.322 \\
... fear of consequences for family members & 1.146 & 0.32 \\
... fear of personalized advertising & 1.112 & 0.357 \\ \hline
\end{tabular}
\caption{Differences in terms of motivations to share genotyping-data, comparison between participants who would share their genotyping data with their DTC provider with participants who would not share their data with their DTC provider.}
\label{tab:motivations2}
\end{table}

Similarly, those people who would share data with their DTC provider under any circumstances are likely to agree more strongly with 
the following motivations for sharing than those who would not share their data with their DTC company.
Those participants who are not willing to share data with their DTC company are likely to agree more strongly with the some motivations 
for not sharing their data when compared to those who would share their data with their DTC company under any circumstances, for an overview of the motivations of both groups, see table \ref{tab:motivations2}.

In the case of curiosity as a motive, there is also a substantial difference between those who would share their data with their DTC company under the condition that it did not share the information and those who would not (mean difference = 1.116 SE = 0.344) as well as those who would share under any circumstances (mean difference = -0.874 SE = 0.182).

In the cases of fear of discrimination and fear of a breach of privacy, substantial differences between all three categories exist. Those who would share their data with their DTC company as long as it did not share the information agree less strongly than those who would not share the data with both fear of discrimination as a motive for not sharing (mean difference = -0.615, SE = 0.345) as well as fear of a breach of privacy (mean difference = -0.668, SE = 0.346). Those who would share their data under any circumstances are even less likely to agree with these motives than those who would share only if their DTC company did not share the information (fear of discrimination: mean value = -0.906, SE = 0.182; breach of privacy: mean difference = -1.203, SE = 0.183).

These survey-results indicate that there is a definite interest in customers of DTC-companies to share their results with other scientists. 

\section*{Discussion}
As the survey was taken online by voluntary participants and was mainly spread in the personal genetics community,
and through social media the results do not necessarily reflect the general population. 
Instead it's expected that it over-represents those people most likely to be interested in a project such as openSNP: 
Customers of DTC genetic testing companies and people with a high interest in biology. 

Due to those limitations it's not possible to get estimates of how many potential users could be acquired for a platform like openSNP. 

\end{document}
